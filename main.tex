\documentclass{my_cv}
%\usepackage[a4paper, total={7in, 9.5in}]{geometry}
\usepackage[a4paper, top=1in, bottom=1in, left=0.5in, right=0.5in]{geometry}
\usepackage{pdfpages}
\usepackage{hyperref}
\input{glyphtounicode}
\pdfgentounicode=1

\begin{document}

\name{William Forrest Drayer}
\contact{Philadelphia, PA}{(330) 618-6527}{drayer587@gmail.com}{\href{https://github.com/drayer587}{github.com/drayer587}}{\href{https://linktr.ee/wfdrayer}{linktr.ee/wfdrayer}}

\section{Education}
\datedsubsection{University of South Florida}{2018--2023}
\begin{itemize}
    \item Ph.D., Chemical Engineering
    \begin{itemize}
        \item Dissertation title: Dynamical Polymer Chain Hetero\-ge\-ne\-i\-ties and Their Impacts on the Glass Transition
    \end{itemize}
\end{itemize}
\datedsubsection{University of Akron}{2013--2018}
\begin{itemize}
    \item B.A., Multidisciplinary Studies
    \begin{itemize}
        \item Primary concentration: Mathematics
        \item Secondary concentration: Physical Chemistry
    \item Minors: Polymer Science and Engineering; Music
    \end{itemize}
\end{itemize}

\section{Work}
\datedsubsection{University of Pennsylvania}{2023--Present}
\paragraph{School of Engineering and Applied Science; Materials Science and Engineering\\}
Post\-doctoral Researcher
\begin{itemize}
    \item Investigating hydroxide and chloride solvation and transport in hydrated, precise polymer membranes
    \item Refined calculations of structure factor to achieve agreement with simulation and experiment
    \item Created a force\-field for simulating new precise polymers with experimental analogues
    \item Supervised four undergraduate researchers
    \item Assisted in Department of Energy grant renewal
\end{itemize}

\datedsubsection{University of South Florida}{2018--2023}
\paragraph{Department of Chemical, Biological, and Materials Engineering\\}
Teaching Associate; Research and Teaching Assistant
\begin{itemize}
    \item Lead instructor for Thermodynamics I (Fall 2022)
    \item Teaching assistant for three semesters of thermodynamics (I and II)
    %; assistance with lecture, assignment, and examination preparation and evaluation, and occasional supplementary lectures.
    \item Simulations and theory investigating the glass transition (particularly in polymers)
    \begin{itemize}
        \item Reinstated and updated in-house job submission software and version control
        \item Extensive use of distributed computing (SLURM, Bash, C++)
        \item Project development and management with Git/Git\-hub (previously Apache Subversion)
        %\item LAMMPS simulation of systems such as bead-spring and all-atom (co-)polymers
        \item Data analysis and visualization using Excel, Python, Julia, MATLAB, and Math\-e\-ma\-ti\-ca
        \item Lab safety and instrument calibration and maintenance (GPC and rheometer)
    \end{itemize}
\end{itemize}

%add USF research and teaching work
\datedsubsection{University of Akron}{2015--2018}
\paragraph{Department of Polymer Engineering;}
Undergraduate Research
\begin{itemize}
    \item Simulating and analyzing molecular dynamics of bead-spring co\-polymers
\end{itemize}

\paragraph{Department of Corrosion Engineering;}
Research Assistant
\begin{itemize}
    \item Developed a class for simulating damage to capsule-embedded coatings in Python (Anaconda)
    \item Simulated coating damage and analyzed self-healing performance for application in anti-corrosive coatings
\end{itemize}

\datedsubsection{NASA Glenn Research Center}{2015 Summer}
\paragraph{Ballistic Impact Lab;}
Research Assistant
\begin{itemize}
    \item Re\-fabricated Hopkins\-on tube for high-speed impact measurements
    \item Selected and installed strain gauge and appropriate adhesive
    \item Prepared ballistic gelatin for impact testing
    \item Operated high-speed impact data collection
    \item Tensile testing on carbon fiber samples
\end{itemize}

\section{Publications and Conferences}
\datedsubsection{Publications}{}
\begin{itemize}
    \item[In Prep] Amorphous Molecular Dynamics Analysis Toolkit: AMDAT; 
    David S. Simmons, William F. Drayer, Pierre Kawak, and others
    \item[2025] Investigating Water Channel Structure and Diffusion in Simulations of Anion Exchange Membranes with Two Precise Polymers; 
    William F. Drayer, Emily M. Duan, James C. Johnson, Karen I. Winey, Amalie L. Frischknecht
    %\item[In Prep] Evidence for Two Mechanisms for Molecular Weight Effects on the Glass Transition Temperature; 
    %William F. Drayer and David S. Simmons
    \item[2025] Effect of Sulfonation Level on the Percolated Morphology and Proton Conductivity of Hydrated Fluorine-Free Copolymers: 
    Experiments \& Simulations; 
    Sol Mi Oh, Victoria Lee, William F. Drayer, Max S. Win, Lindsay F. Jones, Courtney M. Leo, Justin G. Kennemur, Amalie L. Frischknecht, Karen I. Winey;
    DOI: 10.1021/jacsau.5c00218
    \item[2024] Is the Molecular Weight Dependence of the Glass Transition Temperature Caused by a Chain End Effect?;
    William F. Drayer and David S. Simmons; DOI: 10.1021/acs.macromol.4c00419
    \item[2023] Interplay between Dynamic Heterogeneity and Inter\-facial Gradients in a Model Polymer Film; 
    Austin D. Hartley, William F. Drayer, Asieh Ghanekarade, and David S. Simmons; DOI: 10.1063/5.0165650
    \item[2022] Sequence Effects on the Glass Transition of a Model Co\-polymer System; 
    William F. Drayer and David S. Simmons; 
    %Macromolecules 2022 55 (14), 5926-5937; 
    DOI: 10.1021/acs.macromol.2c00664
\end{itemize}

\datedsubsection{Presentations}{}
\begin{itemize}
    \item[2025] Drayer, W., Winey, K., and Frischknecht, A. Channel morphologies and aqueous dynamics in simulations of anion exchange membranes. In APS March Meeting Abstracts (Vol. 2025).
    \item[2024] Drayer, W. and Simmons, D. Evidence for Two Mechanisms Driving Molecular Weight Dependence of the Glass Transition Temperature in Linear Polymers. In APS March Meeting Abstracts (Vol. 2024, pp. D32.00007).
    \item[2023] Drayer, W. and Simmons, D. Mechanistic Origins of Glass Transition Dependence on Molecular Weight in Linear Homo\-polymers. In APS March Meeting Abstracts (Vol. 2023, pp. K23.00005). 
    \item[2022] Drayer, W. and Simmons, D. Computational Insights into the Molecular Origins of the Chain Length Dependence of Polymers' Glass Transition. In APS March Meeting Abstracts (Vol. 2022, pp. Y16.008).
    \item[2021] Drayer, W. and Simmons, D. Sequence Effects on the Glass Transition - Suppression from Block to Alternating Co\-polymers. In APS March Meeting Abstracts (Vol. 2021, pp. S08-005).
    \item[2019] Drayer, W. and Simmons, D. Polymer chain sequence effects on the glass transition. In APS March Meeting Abstracts (Vol. 2019, pp. P54-001).
\end{itemize}

\datedsubsection{Posters}{}
\begin{itemize}
    \item[2024] William F. Drayer, Emily Duan, James Johnson, Karen I. Winey, Amalie L. Frischknecht. 
    Nano\-scale Structure and Water Dynamics in Precisely Quaternized Polymers; Polymer Physics GRC
\end{itemize}

\section{Technical Proficiencies}

\paragraph{Programming Languages:}
\begin{center}
\begin{tabular}{c|c|c|c}
    Julia & Python & Bash & Power\-shell \\
    \hline
    C++ & Mathematica & MATLAB & Java \\
\end{tabular}
\end{center}

\paragraph{Technical Software Experience:}
\begin{center}
\begin{tabular}{c|c|c|c|c}
   Git (Github) & CUDA (C \& Julia) & LAMMPS & SLURM & Anaconda (Python Suite) \\
\end{tabular}
\end{center}

\paragraph{Highlighted Material Characterization Techniques:}

\begin{itemize}
    \item Differential scanning calorimetry (DSC)
    \item Size-exclusion chromatography (SEC)
    \item Rheology
    \item X-ray diffraction (XRD)
\end{itemize}

\section{Awards and Other Work Experience}

\paragraph{Selected Awards:}
\begin{itemize}
    \item USF Outstanding Teaching Assistant Award (2022)
    \item USF Outstanding Departmental Contribution Award (2021)
    \item University of Akron President's List (6 semesters)
    \item University of Akron Dean's List (10 semesters)
    \item Richard L. Waldman, Jr. Scholarship (Fall 2018)
    \item Greater Cleveland Automobile Dealers Association Scholarship Recipient (thrice; 2014-2016)
\end{itemize}

\paragraph{Professional Bassoonist:}
\begin{itemize}
	\item Canton Concert Band, 2012-2018
	\item Alliance Symphony Orchestra, Spring 2012-Spring 2014, Spring 2018
    \item University of Akron, Fall 2013-Spring 2017
	\item Ohio Band Director’s Conference, Spring 2016
	\item Kent State Stark Band, Spring 2012-Spring 2014
\end{itemize}

\paragraph{Laborer:}
\begin{itemize}
    \item Grounds\-keeping (2012-2014)
    \begin{itemize}
        \item Sanctuary Golf Course (bunker maintenance and repair, mowing, edge-trimming, greens\-keeping, etc.)
        \item University of Akron (landscape maintenance)
    \end{itemize}
    \item Farmhand (Summer 2011)
\end{itemize}

\end{document}

\paragraph{Coursework Highlights:}
\begin{center}
\begin{tabular}{c|c}
    Statistical Mechanics & Polymer Chemistry \\
    \hline
    Optics and Scattering Theory & Materials Characterization \\
    \hline
    Parallel Programming (CUDA) & Electrochemical Impedance Spectroscopy \\
    \hline
    Partial Differential Equations & Advanced Calculus \\
\end{tabular}
\end{center}

\paragraph{Primary academic interests:}
\begin{itemize}
    \item infinitesimal calculus
    \item optimization (mathematical and programmatic)
    \item soft matter and polymer dynamics
    \item computational physics
\end{itemize}

\datedsubsection{ABC Limited}{2008-Now}
\workitems
{Developed new product}
{Improved productivity by 20\%}
{Decreased costs by \$10,000}

\section{Skills}
\begin{tabular}{l l l l}
C\# & T-SQL & Javascript & HTML \\
XML & JSON & SOAP & REST
\end{tabular}

Pre-Academic Work Experience
Fall 2014-Spring 2015, Grounds Crew, University of Akron, OH
Summers 2012-2014, Grounds Crew/Greens keeper, Sanctuary Golf Club, North Canton OH
Summer 2011, Farmhand, Maize Valley Farm, Hartville OH

Society Membership
Grace United Church of Christ, consistory member, 2017-2018.
Pi Mu Epsilon member, inducted 2015
